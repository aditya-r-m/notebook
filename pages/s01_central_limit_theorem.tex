hl|tc| de Moivre - Laplace CLT
A special of central limit theorem is the convergence of Binomial to Normal distribution. Binomial distribution can be thought of as sampling distribution with a number of identical independently distributed boolean random variables.
If we start with a Binomial Distribution Symmetric around its mean and standardize it, we can easily show that for a large value of \(n\), the dominating terms of the distribution are identical to the standard Normal using a series of two approximations.
Symmetric Binomial Distribution, Sterling's approximation, and Taylor's series for natual logarithm are defined as follows,
tc| \( P(k) = \frac{1}{2^n} \frac{n!}{k!(n-k)!}, \mu = n/2, \sigma^2 = n/4 \)
tc| \( \ln(n!) = \sum_{i=1}^{n} \ln(i) \approx \int^n_1 \ln(x) dx \approx n \cdot \ln(n) - n \Rightarrow n! \approx n^n e^{-n} \)
tc| \( \ln(1+x) = \sum_{i=1}^{\infty} {(-1)}^{i-1}\frac{x^i}{i} \approx x - \frac{x^2}{2}, 0 \lt x \ll 1 \)
Applying Stirling's Approximation to symmetric Binomial definition, we get the following,
tc| \( \approx \frac{1}{2^n} \frac{n^n e^{-n}}{k^k e^{-k} (n-k)^{n-k} e^{-(n-k)}} \)
tc| \( \Rightarrow \frac{1}{2^n} \frac{n^n}{k^k(n-k)^{n-k}} \)
tc| \( \Rightarrow \left(\frac{n/2}{k}\right)^k \left(\frac{n/2}{n-k}\right)^{n-k} \)
tc| \( \Rightarrow \exp\left\{ k \cdot \ln\left(\frac{n/2}{k}\right) + (n-k) \cdot \ln\left(\frac{n/2}{n-k}\right) \right\} \)
Let \(x = \frac{k - n/2}{\sqrt{n}/2}\) and approximate natural logarithm using Taylor's series,
tc| \( \Rightarrow \exp\left\{\frac{1}{2}\left(-(n + x\sqrt{n}) \cdot \ln\left(1+\frac{x}{\sqrt{n}}\right) + -(n - x\sqrt{n}) \cdot \ln\left(1-\frac{x}{\sqrt{n}}\right) \right)\right\} \)
tc| \( \approx \exp\left\{\frac{1}{2}\left(-(n + x\sqrt{n}) \cdot \left(\frac{x}{\sqrt{n}} - \frac{x^2}{2n}\right) + -(n - x\sqrt{n}) \cdot \left(-\frac{x}{\sqrt{n}} - \frac{x^2}{2n}\right) \right)\right\} \Rightarrow e^{-\frac{x^2}{2}} \)
hl|tc| \( \therefore P(k) \approx e^{-\frac{x^2}{2}} \)
Note that this function is defined by the following core property,
hl|tc| \( \frac{d}{d\alpha}f(\alpha)= -\alpha \cdot f(\alpha), \)
If we have non-zero mean and non-unit variance, the following generalized version holds,
hl|tc| \( \frac{d}{d\alpha}f(\alpha)= -\frac{\alpha-\mu}{\sigma^2} \cdot f(\alpha) \)
We can show that the same holds for a Binomial distribution as follows,
tc| \( P(k+1) - P(k) = \frac{n!}{(k+1)!(n-(k+1))!} - \frac{n!}{k!(n-k)!} \)
tc| \( \Rightarrow P(k)\left(\frac{n-k}{k+1} - 1\right) \Rightarrow P(k)\frac{n-2k-1}{k+1} \)
tc| \( \Rightarrow P(k)\frac{n-2(n/2 + x\sqrt{n}/2)-1}{n/2 + x\sqrt{n}/2 +1} \Rightarrow -P(k)\frac{x\sqrt{n} + 1}{n/2 + x\sqrt{n}/2 + 1} \approx -\frac{x}{\sqrt{n}/2}P(k)\frac{xn/2}{xn/2} \)
hl|tc| \( \therefore P(k+1)-P(k) \rightarrow -\frac{k-n/2}{n/4}P(k) \)
hl|tc| Lindeberg-Levy CLT
The general CLT focuses on Samples taken from any probability distribution with finite mean and variance. By analysing characterstic function or moment-generating function, it can be seen that the structure approaches the Normal Distribution.
Note that the Characterstic function of a distribution is analogous to Fourier Transform and MGF is a variant of the same which abstracts away the process of computing \(n^{th}\) moment of the distribution.
Assuming \({X_1,X_2,..X_n}\) are independent identical random variables with finite mean \(\mu\) and variance \(\sigma^2\),
tc| \( Z_n = \sum_{i=1}^{n}\frac{X_i - \mu}{\sqrt{n}\sigma} = \sum_{i=1}^{n}\frac{Y_i}{\sqrt{n}}, E[Y_i] = 0, V[Y_i] = 1 \)
Focusing on the characterstic function of \(Z_n\),
tc| \( \phi_{Z_n}(t) = \phi_{\sum_{i=1}^{n}Y_i/\sqrt{n}}(t) = \phi_{Y_1}(t/\sqrt{n})^n \)
tc| \( \Rightarrow \phi_{Z_n}(t) = (1 + E[Y_1]\cdot it/\sqrt{n} - V[Y_1]\cdot t^2/2n + \epsilon)^n \)
hl|tc| \( \phi_{Z_n}(t) \approx e^{\frac{-t^2}{2}} \)
Focusing on the characterstic function of standard normal distribution,
tc| \( \phi_{N(0,1)} = \int \frac{1}{\sqrt{2\pi}} e^{\frac{-x^2}{2}} e^{itx} dx = \frac{1}{\sqrt{2\pi}} \int e^{itx  - \frac{x^2}{2}} dx = \frac{1}{\sqrt{2\pi}} \int e^{-\frac{1}{2} (x^2 - 2itx + (it)^2)} dx \)
tc| \( \Rightarrow \phi_{N(0,1)} = e^{\frac{-t^2}{2}} \frac{1}{\sqrt{2\pi}} \int e^{-\frac{1}{2} (x - it)^2} dx \)
hl|tc| \( \phi_{N(0,1)} = e^{\frac{-t^2}{2}} \)
hl|tc| \( \therefore Z_n \rightarrow N(0, 1) \)